% Paul Engelke (u13093500)

\subsubsection*{Notification A}
Below are some of the issues that could be identified in the module provided by Notifications A, in terms of reliability.
\begin{itemize}
    \item \textbf{Function Return Values:}\newline
     All functions available in the module's API should return some indication of success.
     However, in the case of this module, all the functions that return a value, only return true (proof of which can be found on lines 137, 169, 215, 260, 305). This means that 
     regardless of what the outcome of the function's execution is, the caller will only perceive it as 
     having succeeded --- which is an issue of reliability: if an error occurs and needs to be handled by the system, 
     nothing can be done because the function does not notify the system of erroneous execution.
     \item \textbf{Callback Usage:}\newline
     A follow-on of the previous point is the usage of callbacks.
     No provision has been made for the use of callback functions and therefore any 
     information provided by asynchronous execution cannot be conveyed to the caller. So any return value, regardless of whether it may be true or false, cannot be trusted as the asynchronous part of the function will most likely not have been completed before it returns to the caller, i.e. the return value is unreliable.\newline
     Below is a listing of the function signatures that require callbacks, but do not make provision for them:
\end{itemize}
\begin{lstlisting}
     line 94  : notification.notifyRegistration(jsonObject)
     line 140 : notification.notifyDeregistration(jsonObject)
     line 172 : notification.notifyNewPost(jsonObject)
     line 218 : notification.notifyDeletedThread(jsonObject)
     line 263 : notification.notifyMovedThread(jsonObject)
     line 308 : notification.appraisalRegister(jsonObject)
     line 321 : notification.appraisalDeregister(jsonObject)
     line 336 : notification.appraisalNotify(jsonObject)
     line 390 : notification.sendNotification(jsonObject)
\end{lstlisting}
   
\begin{itemize}
\item \textbf{Logging to Console:}\newline
     In the cases where errors occur and in some cases where successful execution occurs, information about the success or failure is lost to the system as it is only logged to the console(not reliable as a persistence tool) instead of being returned in some form that the caller can interpret. This is another issue of reliability as the system cannot determine whether or not a user will now receive notifications or if a user has been deregistered from receiving notifications, etc.
     An example of the above can be found starting on line 321: 
\end{itemize}
\begin{lstlisting}
    notification.appraisalDeregister = function (jsonObj) {
        appraisals.remove({
            notification_StudentID: jsonObj.studentID,
            notification_AppraisalType: jsonObj.appraisalType
        }, function (err) {
           if (err != null)
             console.log(err);
           else
             console.log("user has been removed from table");
        });
    };
\end{lstlisting}

\subsubsection*{Notification B}
Below are discussed some of the issues identified, in terms of reliability, for the Notifications B module.
\begin{itemize}
	\item \textbf{Function Return Values:}\newline
	In many cases, no values are returned at all, which forces the system to assume that the call executed correctly. This causes the system to be unreliable as there is no guarantee that the function call succeeded or not.
	\item \textbf{Callback Usage}\newline
	In most cases, like Notifications A, no provision has been made for callback functions to asynchronous function calls. This results in no means of obtaining reliable result information from the functions. Below is a list of the files containing the functions that fail to make provision for callback functions where they are needed:
		\begin{lstlisting}
		AppraisalNotifyMe.js
		DailyNotif.JS
		DeleteNotif.js
		StandardNotification.js
		\end{lstlisting}
	\item \textbf{Logging to Console:}\newline
\end{itemize}
\subsubsection*{Remarks}
[Remarks go here...]